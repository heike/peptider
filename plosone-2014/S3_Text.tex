{\bf Simulation Results.}
\hh{A simulation scenario is used to validate results from the theoretical framework in practice.

We draw  a peptide sequence of length k from amino acids that are represented in their frequency  by the number of codons specified by the respective library scheme. For the example of a NNK/S library scheme, this means amino acids S, L, and R are selected with probability 3/32, amino acids A, G, P, T, and V are selected with probability 2/32, while all other amino acids and a stop codon are selected with probability 1/32. 
For k = 7 a sample of these peptide sequences might look like: {\tt KFRTVIR, RR*DISY, *LWAEPP, YAYEN*S, RMRQFWP, LYHPVIT, VNMMRHS, SEGGGRG, NYS*RT*, RA*LTAL}.
A library is made up of a sample of N of these sequences. Peptide diversity of the library is then determined by first removing all invalid peptide sequences (the ones that include the stop codon *), and secondly removing all duplicate peptide sequences. The number of peptide sequences left gives the diversity of the library. 

The simulation was set up to sample k-peptide libraries of sizes between 10\textsuperscript{4} and 10\textsuperscript{6}, for schemes NNN, NNB, NNK/S, and 20/20. k is chosen as a length between 7 and 10. For each of these scenarios, 100 libraries are sampled, diversity is determined for each one of them. The results are shown in the \nameref{fig:simulation} and \nameref{tab:simulation}. The \nameref{fig:simulation} shows histograms of sampled library diversity of 8-peptide libraries of size N = 10\textsuperscript{5} under all four library schemes. NNN results in the lowest diversity number, while 20/20 libraries are almost perfect -- only two libraries had two duplicate peptides, while another 13 libraries had a single duplicate. Here, the NNB scheme has a  slightly higher diversity than the NNK/S scheme, because of the smaller initial loss under NNB and the relatively small library size (N = 10\textsuperscript{5}), which is much lower than half the number of possible peptides, at which point diversity would tip in favor of NNK/S.
}
\hh{The \nameref{tab:simulation} and \nameref{tab:variance} give an overview of observed versus expected library diversity and the variance for each of the scenarios. It is obvious, that both expected and observed values are extremely close to each other. }
